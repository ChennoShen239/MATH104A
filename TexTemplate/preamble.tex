% Copyright (c) Brandon Pacewic
% SPDX-License-Identifier: MIT

% Fix section numbering
\setcounter{secnumdepth}{3}
\renewcommand{\thesection}{\arabic{section}}
\renewcommand{\thesubsection}{\thesection.\arabic{subsection}}

\usepackage[tmargin=2cm,rmargin=1in,lmargin=1in,margin=0.85in,bmargin=2cm,footskip=.2in]{geometry}
\usepackage{amsmath,amsfonts,amsthm,amssymb,mathtools}
\usepackage[varbb]{newpxmath}
\usepackage{xfrac}
\usepackage[makeroom]{cancel}
\usepackage{mathtools}
\usepackage{bookmark}
\usepackage{enumitem}
\usepackage{hyperref,theoremref}
\hypersetup{
	pdftitle={Assignment},
	colorlinks=true, linkcolor=doc!90,
	bookmarksnumbered=true,
	bookmarksopen=true
}
\usepackage[most,many,breakable]{tcolorbox}
\usepackage{xcolor}
\usepackage{varwidth}
\usepackage{etoolbox}
\usepackage{nameref}
\usepackage{multicol,array}
\usepackage{tikz-cd}
\usepackage[ruled,vlined,linesnumbered]{algorithm2e}
\usepackage{import}
\usepackage{xifthen}
\usepackage{pdfpages}
\usepackage{transparent}

% Graphs
\usepackage{pgfplots}
\pgfplotsset{width=10cm,compat=1.9}
\usepgfplotslibrary{fillbetween}

\newcommand\mycommfont[1]{\footnotesize\ttfamily\textcolor{blue}{#1}}
\SetCommentSty{mycommfont}
\newcommand{\incfig}[1]{%
    \def\svgwidth{\columnwidth}
    \import{./figures/}{#1.pdf_tex}
}

\usepackage{tikzsymbols}
\renewcommand\qedsymbol{$\Laughey$}

% Theorem-like environment colors
\definecolor{theoremcolor}{HTML}{F2F2F9}
\definecolor{examplecolor}{HTML}{F2FBF8}
\definecolor{definitioncolor}{HTML}{F5F0FF}
\definecolor{exercisecolor}{HTML}{F2FBF8}
\definecolor{notecolor}{HTML}{FFF0E6}
\definecolor{problemcolor}{HTML}{FFF2F2}
\definecolor{solutioncolor}{HTML}{F2F8FF}

% Common settings for all theorem-like environments
\tcbuselibrary{theorems,skins,hooks}
\tcbset{
    enhanced jigsaw,
    breakable,
    boxrule=0.4pt,
    leftrule=2pt,
    left=5mm,
    right=2mm,
    top=2mm,
    bottom=2mm,
    sharp corners,
    fonttitle=\bfseries\sffamily,
    description font=\mdseries,
    separator sign none,
    before title={\hspace*{-4mm}},
    before upper={\parindent0pt},
    after title={\par\smallskip},
    attach title to upper
}

% Theorem environment
\newtcolorbox[auto counter,number within=section]{theorembox}[2][]{
    colback=theoremcolor,
    colframe=blue!70!black,
    colbacktitle=theoremcolor,
    coltitle=blue!70!black,
    fonttitle=\bfseries\sffamily,
    title=Theorem~\thetcbcounter: #2,
    #1
}

% Example environment
\newtcolorbox[auto counter,number within=section]{examplebox}[2][]{
    colback=examplecolor,
    colframe=green!70!black,
    colbacktitle=examplecolor,
    coltitle=green!70!black,
    fonttitle=\bfseries\sffamily,
    title=Example~\thetcbcounter: #2,
    #1
}

% Definition environment
\newtcolorbox[auto counter,number within=section]{definitionbox}[2][]{
    colback=definitioncolor,
    colframe=violet!70!black,
    colbacktitle=definitioncolor,
    coltitle=violet!70!black,
    fonttitle=\bfseries\sffamily,
    title=Definition~\thetcbcounter: #2,
    #1
}

% Exercise environment
\newtcolorbox[auto counter,number within=section]{exercisebox}[2][]{
    colback=exercisecolor,
    colframe=teal!70!black,
    colbacktitle=exercisecolor,
    coltitle=teal!70!black,
    fonttitle=\bfseries\sffamily,
    title=Exercise~\thetcbcounter: #2,
    #1
}

% Note environment
\newtcolorbox{note}{
    colback=notecolor,
    colframe=orange!70!black,
    colbacktitle=notecolor,
    coltitle=orange!70!black,
    title=Note,
    fonttitle=\bfseries\sffamily
}

% Problem environment
\newtcolorbox[auto counter,number within=section]{problembox}[2][]{
    colback=problemcolor,
    colframe=red!70!black,
    colbacktitle=problemcolor,
    coltitle=red!70!black,
    fonttitle=\bfseries\sffamily,
    title=Problem~\thetcbcounter: #2,
    #1
}

% Solution environment
\newtcolorbox{solution}{
    colback=solutioncolor,
    colframe=blue!50!black,
    colbacktitle=solutioncolor,
    coltitle=blue!50!black,
    title=Solution,
    fonttitle=\bfseries\sffamily
}

% Create aliases for the environments
\newenvironment{Theorem}[2]
  {\begin{theorembox}[label=#2]{#1}}
  {\end{theorembox}}

\newenvironment{Example}[2]
  {\begin{examplebox}[label=#2]{#1}}
  {\end{examplebox}}

\newenvironment{Definition}[2]
  {\begin{definitionbox}[label=#2]{#1}}
  {\end{definitionbox}}

\newenvironment{Exercise}[2]
  {\begin{exercisebox}[label=#2]{#1}}
  {\end{exercisebox}}

\newenvironment{Problem}[2]
  {\begin{problembox}[label=#2]{#1}}
  {\end{problembox}}
