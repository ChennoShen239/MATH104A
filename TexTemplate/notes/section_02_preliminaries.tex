\section{Preliminaries on Sets, Mappings, and Relations}

\begin{definitionbox}{Set Membership}
Let $A$ be a set. For an element $x$:
\begin{itemize}
    \item $x \in A$ denotes $x$ is a member of $A$
    \item $x \notin A$ denotes $x$ is not a member of $A$
\end{itemize}
Sets are typically denoted by $\{x \mid \text{condition}\}$, representing all elements satisfying the condition.
\end{definitionbox}

\begin{definitionbox}{Set Equality and Subset}
Two sets are equal if and only if they have exactly the same members.

For sets $A$ and $B$:
\begin{itemize}
    \item $A \subseteq B$ means every member of $A$ is a member of $B$
    \item $A$ is a proper subset of $B$ if $A \subseteq B$ and $A \neq B$
\end{itemize}
\end{definitionbox}

\begin{definitionbox}{Set Operations}
For sets $A$ and $B$:
\begin{itemize}
    \item Union: $A \cup B = \{x \mid x \in A \text{ or } x \in B\}$
    \item Intersection: $A \cap B = \{x \mid x \in A \text{ and } x \in B\}$
    \item Complement: $B \setminus A = \{x \mid x \in B \text{ and } x \notin A\}$
\end{itemize}
\end{definitionbox}

\begin{theorembox}{De Morgan's Laws}
For a family of sets $\mathcal{F}$ and a reference set $X$:
\begin{align*}
    X \setminus \left(\bigcup_{F \in \mathcal{F}} F\right) &= \bigcap_{F \in \mathcal{F}} (X \setminus F) \\
    X \setminus \left(\bigcap_{F \in \mathcal{F}} F\right) &= \bigcup_{F \in \mathcal{F}} (X \setminus F)
\end{align*}
\end{theorembox}

\subsection{Mappings and Functions}

\begin{definitionbox}{Mapping}
A mapping (or function) $f: A \to B$ is a correspondence that assigns to each $x \in A$ a unique $y \in B$, denoted $f(x)$.
\begin{itemize}
    \item Domain: The set $A$
    \item Image/Range: $f(A) = \{b \in B \mid \exists a \in A, f(a) = b\}$
\end{itemize}
\end{definitionbox}

\begin{definitionbox}{Function Types}
\begin{itemize}
    \item Onto (Surjective): $f(A) = B$
    \item One-to-One (Injective): $f(a_1) = f(a_2) \implies a_1 = a_2$
    \item Invertible (Bijective): Both one-to-one and onto
\end{itemize}
\end{definitionbox}

\begin{definitionbox}{Function Composition}
For mappings $f: A \to B$ and $g: B \to C$, the composition $g \circ f: A \to C$ is defined by $(g \circ f)(x) = g(f(x))$.
\end{definitionbox}

\subsection{Relations}

\begin{definitionbox}{Relation}
A relation $R$ on a set $X$ is a subset of $X \times X$. 
\begin{itemize}
    \item Reflexive: $\forall x \in X, xRx$
    \item Symmetric: $x R y \implies y R x$
    \item Transitive: $x R y \text{ and } y R z \implies x R z$
\end{itemize}
\end{definitionbox}

\begin{definitionbox}{Equivalence Relation}
An equivalence relation is a relation that is reflexive, symmetric, and transitive.

For an equivalence relation $R$ on $X$:
\begin{itemize}
    \item Equivalence class of $x$: $[x]_R = \{y \in X \mid xRy\}$
    \item Quotient set: $X/R = \{[x]_R \mid x \in X\}$
\end{itemize}
\end{definitionbox}

\subsection{Axiom of Choice and Zorn's Lemma}

\begin{definitionbox}{Choice Function}
A choice function on a family $\mathcal{F}$ of non-empty sets is a function $f: \mathcal{F} \to \bigcup_{F \in \mathcal{F}} F$ such that $f(F) \in F$ for all $F \in \mathcal{F}$.
\end{definitionbox}

\begin{theorembox}{Axiom of Choice}
For any non-empty collection of non-empty sets, there exists a choice function.
\end{theorembox}

\begin{definitionbox}{Partial Ordering}
A relation $\preceq$ on a set $X$ is a partial ordering if it is:
\begin{itemize}
    \item Reflexive: $x \preceq x$
    \item Antisymmetric: $x \preceq y \text{ and } y \preceq x \implies x = y$
    \item Transitive: $x \preceq y \text{ and } y \preceq z \implies x \preceq z$
\end{itemize}
\end{definitionbox}

\begin{theorembox}{Zorn's Lemma}
If a partially ordered set has an upper bound for every totally ordered subset, then it contains a maximal element.
\end{theorembox}

\begin{note}
Zorn's Lemma is equivalent to the Axiom of Choice and will be crucial in proving several important theorems in advanced mathematics.
\end{note}

\subsection{Cardinality and Set Comparisons}

\begin{definitionbox}{Cardinality}
Two sets $A$ and $B$ are said to be equipotent (or have the same cardinality) if there exists a bijective mapping $f: A \to B$. 

Key properties:
\begin{itemize}
    \item Equipotence is an equivalence relation on the collection of sets
    \item $|A| = |B|$ if and only if there exists a bijection between $A$ and $B$
    \item Finite sets have cardinality equal to the number of their elements
\end{itemize}
\end{definitionbox}

\begin{theorembox}{Cantor-Bernstein Theorem}
If there exist injective functions $f: A \to B$ and $g: B \to A$, then there exists a bijection between $A$ and $B$. 

Symbolically, if $|A| \leq |B|$ and $|B| \leq |A|$, then $|A| = |B|$.
\end{theorembox}
