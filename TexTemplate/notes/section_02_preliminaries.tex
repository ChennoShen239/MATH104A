\section{Preliminaries on Sets, Mappings, and Relations}

\begin{note}
In these preliminaries, we describe some notions regarding sets, mappings, and relations that will be used throughout the book. Our purpose is descriptive and the arguments given are directed toward plausibility and understanding rather than rigorous proof based on an axiomatic basis for set theory.
\end{note}

\subsection{Unions and Intersections of Sets}

\begin{definitionbox}{Set Membership}
For a set $A$, the membership of an element $x$ in $A$ is denoted by $x \in A$ and the nonmembership of $x$ in $A$ is denoted by $x \notin A$. 

Sets are often denoted by $\{x \mid \text{statement about } x\}$, representing all elements for which the statement is true. Two sets are the same provided they have the same members.
\end{definitionbox}

\begin{definitionbox}{Subset}
Let $A$ and $B$ be sets. $A$ is a subset of $B$, denoted by $A \subseteq B$, provided each member of $A$ is a member of $B$. 

A subset $A$ of $B$ is called a proper subset of $B$ provided $A \neq B$.
\end{definitionbox}

\begin{definitionbox}{Set Operations}
For sets $A$ and $B$:
\begin{itemize}
    \item Union: $A \cup B = \{x \mid x \in A \text{ or } x \in B\}$ (in the nonexclusive sense)
    \item Intersection: $A \cap B = \{x \mid x \in A \text{ and } x \in B\}$
    \item Complement: $B \setminus A = \{x \mid x \in B, x \notin A\}$
\end{itemize}

If all sets are subsets of a reference set $X$, $X \setminus A$ is simply called the complement of $A$.
\end{definitionbox}

\begin{definitionbox}{Special Sets}
\begin{itemize}
    \item Empty set: A set with no members, denoted by $\textcircled{1}$
    \item Non-empty set: A set that is not equal to the empty set
    \item Singleton set: A set with a single member
    \item Power set: For a set $X$, $\mathcal{P}(X)$ or $2^X$ is the set of all subsets of $X$
\end{itemize}
\end{definitionbox}

\begin{definitionbox}{Generalized Set Operations}
Let $\mathcal{F}$ be a collection of sets:
\begin{itemize}
    \item Union: $\bigcup_{F\in\mathcal{F}}F$ is the set of points that belong to at least one set in $\mathcal{F}$
    \item Intersection: $\bigcap_{F\in\mathcal{F}}F$ is the set of points that belong to every set in $\mathcal{F}$
\end{itemize}

A collection of sets $\mathcal{F}$ is disjoint if the intersection of any two distinct sets in $\mathcal{F}$ is empty.
\end{definitionbox}

\begin{theorembox}{De Morgan's Identities}
For a family $\mathcal{F}$ of sets and a reference set $X$:
\begin{align*}
    X \setminus \left[\bigcup_{F\in\mathcal{F}}F\right] &= \bigcap_{F\in\mathcal{F}}[X \setminus F] \\
    X \setminus \left[\bigcap_{F\in\mathcal{F}}F\right] &= \bigcup_{F\in\mathcal{F}}[X \setminus F]
\end{align*}
\end{theorembox}

\subsection{Mappings between Sets}

\begin{definitionbox}{Mapping/Function}
Given sets $A$ and $B$, a mapping (or function) $f: A \to B$ is a correspondence that assigns to each member of $A$ a member of $B$. 

For each $x \in A$, $f(x)$ denotes the member of $B$ to which $x$ is assigned.
\end{definitionbox}

\begin{definitionbox}{Image and Domain}
For a mapping $f: A \to B$ and a subset $A' \subseteq A$:
\begin{itemize}
    \item Domain: The set $A$
    \item Image of $A'$: $f(A') = \{b \mid b = f(a) \text{ for some } a \in A'\}$
    \item Range/Image of $f$: $f(A)$
\end{itemize}
\end{definitionbox}

\begin{definitionbox}{Function Types}
\begin{itemize}
    \item Onto (Surjective): $f(A) = B$
    \item One-to-One (Injective): For each $b \in f(A)$, there is exactly one $a \in A$ such that $b = f(a)$
    \item Invertible: Both one-to-one and onto, establishing a one-to-one correspondence
\end{itemize}
\end{definitionbox}

\begin{definitionbox}{Inverse Mapping}
For an invertible mapping $f: A \to B$:
\begin{itemize}
    \item $f^{-1}(b)$ is the unique $a \in A$ such that $f(a) = b$
    \item $f^{-1}: B \to A$ is the inverse mapping
\end{itemize}

Two sets $A$ and $B$ are equipotent if there exists an invertible mapping from $A$ to $B$.
\end{definitionbox}

\begin{definitionbox}{Function Composition}
For mappings $f: A \to B$ and $g: C \to D$ where $f(A) \subseteq C$:
\begin{itemize}
    \item Composition $g \circ f: A \to D$ defined by $[g \circ f](x) = g(f(x))$
    \item The composition of invertible mappings is invertible
\end{itemize}
\end{definitionbox}

\begin{definitionbox}{Inverse Image}
For a mapping $f: A \to B$ and a set $E \subseteq B$:
\begin{itemize}
    \item $f^{-1}(E) = \{a \in A \mid f(a) \in E\}$
    \item Useful properties:
    \begin{align*}
        f^{-1}(E_1 \cup E_2) &= f^{-1}(E_1) \cup f^{-1}(E_2) \\
        f^{-1}(E_1 \cap E_2) &= f^{-1}(E_1) \cap f^{-1}(E_2) \\
        f^{-1}(E_1 \setminus E_2) &= f^{-1}(E_1) \setminus f^{-1}(E_2)
    \end{align*}
\end{itemize}
\end{definitionbox}

\begin{definitionbox}{Function Restriction}
For a mapping $f: A \to B$ and $A' \subseteq A$, the restriction of $f$ to $A'$, denoted $f\big|_{A'}$, is the mapping from $A'$ to $B$ which assigns $f(x)$ to each $x \in A'$.
\end{definitionbox}

\subsection{Equivalence Relations, Axiom of Choice, and Zorn's Lemma}

\begin{definitionbox}{Cartesian Product}
For non-empty sets $A$ and $B$, the Cartesian product $A \times B$ is the collection of all ordered pairs $(a,b)$ where $a \in A$ and $b \in B$, with $(a,b) = (a',b')$ if and only if $a = a'$ and $b = b'$.
\end{definitionbox}

\begin{definitionbox}{Relation}
For a non-empty set $X$, a relation $R$ on $X$ is a subset of $X \times X$. 
\begin{itemize}
    \item Reflexive: $xRx$ for all $x \in X$
    \item Symmetric: $xRy \implies yRx$
    \item Transitive: $xRy$ and $yRz \implies xRz$
\end{itemize}
\end{definitionbox}

\begin{definitionbox}{Equivalence Relation}
An equivalence relation $R$ on a set $X$ is a relation that is reflexive, symmetric, and transitive.

For $x \in X$, the equivalence class $R_x = \{x' \in X \mid xRx'\}$. 
The collection of equivalence classes is denoted $X/R$.
\end{definitionbox}

\begin{definitionbox}{Choice Function}
Let $\mathcal{F}$ be a non-empty family of non-empty sets. A choice function $f$ on $\mathcal{F}$ is a function $f: \mathcal{F} \to \bigcup_{F\in\mathcal{F}}F$ such that $f(F) \in F$ for each $F \in \mathcal{F}$.
\end{definitionbox}

\begin{theorembox}{Axiom of Choice}
For any non-empty collection of non-empty sets, there exists a choice function.
\end{theorembox}

\begin{definitionbox}{Partial Ordering}
A relation $R$ on a non-empty set $X$ is a partial ordering if it is:
\begin{itemize}
    \item Reflexive: $x R x$
    \item Antisymmetric: If $xRy$ and $yRx$, then $x = y$
    \item Transitive: If $xRy$ and $yRz$, then $xRz$
\end{itemize}
\end{definitionbox}

\begin{theorembox}{Zorn's Lemma}
Let $X$ be a partially ordered set where every totally ordered subset has an upper bound. Then $X$ has a maximal member.
\end{theorembox}

\begin{note}
Zorn's Lemma is equivalent to the Axiom of Choice and will be crucial in proving important theorems such as the Hahn-Banach Theorem, the Tychonoff Product Theorem, and the Krein-Milman Theorem.
\end{note}

\begin{definitionbox}{Generalized Cartesian Product}
For a collection of sets $\{E_{\lambda}\}_{\lambda\in\Lambda}$ parametrized by $\Lambda$, the Cartesian product $\prod_{\lambda\in\Lambda}E_{\lambda}$ is the set of functions $f: \Lambda \to \bigcup_{\lambda\in\Lambda}E_{\lambda}$ such that $f(\lambda) \in E_{\lambda}$ for each $\lambda \in \Lambda$.
\end{definitionbox}

\begin{note}
The Axiom of Choice is equivalent to the assertion that the Cartesian product of a non-empty family of non-empty sets is non-empty.
\end{note}
