% Math Class Notes
% Prepared by Chen Gao
% SPDX-License-Identifier: MIT

\documentclass[12pt,a4paper]{article}

% Copyright (c) Brandon Pacewic
% SPDX-License-Identifier: MIT

% Fix section numbering
\setcounter{secnumdepth}{3}
\renewcommand{\thesection}{\arabic{section}}
\renewcommand{\thesubsection}{\thesection.\arabic{subsection}}

\usepackage[tmargin=2cm,rmargin=1in,lmargin=1in,margin=0.85in,bmargin=2cm,footskip=.2in]{geometry}
\usepackage{amsmath,amsfonts,amsthm,amssymb,mathtools}
\usepackage[varbb]{newpxmath}
\usepackage{xfrac}
\usepackage[makeroom]{cancel}
\usepackage{mathtools}
\usepackage{bookmark}
\usepackage{enumitem}
\usepackage{hyperref,theoremref}
\hypersetup{
	pdftitle={Assignment},
	colorlinks=true, linkcolor=doc!90,
	bookmarksnumbered=true,
	bookmarksopen=true
}
\usepackage[most,many,breakable]{tcolorbox}
\usepackage{xcolor}
\usepackage{varwidth}
\usepackage{etoolbox}
\usepackage{nameref}
\usepackage{multicol,array}
\usepackage{tikz-cd}
\usepackage[ruled,vlined,linesnumbered]{algorithm2e}
\usepackage{import}
\usepackage{xifthen}
\usepackage{pdfpages}
\usepackage{transparent}

% Graphs
\usepackage{pgfplots}
\pgfplotsset{width=10cm,compat=1.9}
\usepgfplotslibrary{fillbetween}

\newcommand\mycommfont[1]{\footnotesize\ttfamily\textcolor{blue}{#1}}
\SetCommentSty{mycommfont}
\newcommand{\incfig}[1]{%
    \def\svgwidth{\columnwidth}
    \import{./figures/}{#1.pdf_tex}
}

\usepackage{tikzsymbols}
\renewcommand\qedsymbol{$\Laughey$}

% Theorem-like environment colors
\definecolor{theoremcolor}{HTML}{F2F2F9}
\definecolor{examplecolor}{HTML}{F2FBF8}
\definecolor{definitioncolor}{HTML}{F5F0FF}
\definecolor{exercisecolor}{HTML}{F2FBF8}
\definecolor{notecolor}{HTML}{FFF0E6}
\definecolor{problemcolor}{HTML}{FFF2F2}
\definecolor{solutioncolor}{HTML}{F2F8FF}

% Common settings for all theorem-like environments
\tcbuselibrary{theorems,skins,hooks}
\tcbset{
    enhanced jigsaw,
    breakable,
    boxrule=0.4pt,
    leftrule=2pt,
    left=5mm,
    right=2mm,
    top=2mm,
    bottom=2mm,
    sharp corners,
    fonttitle=\bfseries\sffamily,
    description font=\mdseries,
    separator sign none,
    before title={\hspace*{-4mm}},
    before upper={\parindent0pt},
    after title={\par\smallskip},
    attach title to upper
}

% Theorem environment
\newtcolorbox[auto counter,number within=section]{theorembox}[2][]{
    colback=theoremcolor,
    colframe=blue!70!black,
    colbacktitle=theoremcolor,
    coltitle=blue!70!black,
    fonttitle=\bfseries\sffamily,
    title=Theorem~\thetcbcounter: #2,
    #1
}

% Example environment
\newtcolorbox[auto counter,number within=section]{examplebox}[2][]{
    colback=examplecolor,
    colframe=green!70!black,
    colbacktitle=examplecolor,
    coltitle=green!70!black,
    fonttitle=\bfseries\sffamily,
    title=Example~\thetcbcounter: #2,
    #1
}

% Definition environment
\newtcolorbox[auto counter,number within=section]{definitionbox}[2][]{
    colback=definitioncolor,
    colframe=violet!70!black,
    colbacktitle=definitioncolor,
    coltitle=violet!70!black,
    fonttitle=\bfseries\sffamily,
    title=Definition~\thetcbcounter: #2,
    #1
}

% Exercise environment
\newtcolorbox[auto counter,number within=section]{exercisebox}[2][]{
    colback=exercisecolor,
    colframe=teal!70!black,
    colbacktitle=exercisecolor,
    coltitle=teal!70!black,
    fonttitle=\bfseries\sffamily,
    title=Exercise~\thetcbcounter: #2,
    #1
}

% Note environment
\newtcolorbox{note}{
    colback=notecolor,
    colframe=orange!70!black,
    colbacktitle=notecolor,
    coltitle=orange!70!black,
    title=Note,
    fonttitle=\bfseries\sffamily
}

% Problem environment
\newtcolorbox[auto counter,number within=section]{problembox}[2][]{
    colback=problemcolor,
    colframe=red!70!black,
    colbacktitle=problemcolor,
    coltitle=red!70!black,
    fonttitle=\bfseries\sffamily,
    title=Problem~\thetcbcounter: #2,
    #1
}

% Solution environment
\newtcolorbox{solution}{
    colback=solutioncolor,
    colframe=blue!50!black,
    colbacktitle=solutioncolor,
    coltitle=blue!50!black,
    title=Solution,
    fonttitle=\bfseries\sffamily
}

% Create aliases for the environments
\newenvironment{Theorem}[2]
  {\begin{theorembox}[label=#2]{#1}}
  {\end{theorembox}}

\newenvironment{Example}[2]
  {\begin{examplebox}[label=#2]{#1}}
  {\end{examplebox}}

\newenvironment{Definition}[2]
  {\begin{definitionbox}[label=#2]{#1}}
  {\end{definitionbox}}

\newenvironment{Exercise}[2]
  {\begin{exercisebox}[label=#2]{#1}}
  {\end{exercisebox}}

\newenvironment{Problem}[2]
  {\begin{problembox}[label=#2]{#1}}
  {\end{problembox}}

\usepackage{libertine}
% Hyperref configuration for better TOC and cross-referencing
\hypersetup{
    colorlinks=true,
    linkcolor=blue,
    filecolor=magenta,      
    urlcolor=cyan,
    pdftitle={Mathematical Analysis Class Notes},
    pdfauthor={Chen Gao},
    pdfpagemode=FullScreen,
}

\title{Introduction to Analysis}
\author{Chen Gao}
\date{Spring 2025}

\begin{document}

% Vertical centering for titlepage
\newgeometry{margin=1in}
\begin{titlepage}
    \begin{center}
        \vspace*{\fill}
        
        \huge\textbf{Introduction to Analysis}
        
        \vspace{1cm}
        
        \Large{Chen Gao}
        
        \vspace{1cm}
        
        \large{Spring 2025}
        
        \vspace{2cm}
        
        \begin{tabular}{ll}
        \textbf{Course:} & MATH 104A \\
        \textbf{Instructor:} & Professor [who] \\
        \textbf{Institution:} & University of California, Berkeley \\
        \textbf{Semester:} & Spring \the\year
        \end{tabular}
        
        \vspace*{\fill}
    \end{center}
\end{titlepage}
\restoregeometry

% Ensure proper TOC generation
\newpage
\tableofcontents
\newpage

\section{Course Overview}
\begin{note}
    These notes compile key concepts, theorems, and insights from the Mathematical Analysis course. 
    They serve as a comprehensive reference for understanding advanced mathematical principles.
\end{note}

\section{Preliminaries on Sets, Mappings, and Relations}

\begin{definitionbox}{Set Membership}
Let $A$ be a set. For an element $x$:
\begin{itemize}
    \item $x \in A$ denotes $x$ is a member of $A$
    \item $x \notin A$ denotes $x$ is not a member of $A$
\end{itemize}
Sets are typically denoted by $\{x \mid \text{condition}\}$, representing all elements satisfying the condition.
\end{definitionbox}

\begin{definitionbox}{Set Equality and Subset}
Two sets are equal if and only if they have exactly the same members.

For sets $A$ and $B$:
\begin{itemize}
    \item $A \subseteq B$ means every member of $A$ is a member of $B$
    \item $A$ is a proper subset of $B$ if $A \subseteq B$ and $A \neq B$
\end{itemize}
\end{definitionbox}

\begin{definitionbox}{Set Operations}
For sets $A$ and $B$:
\begin{itemize}
    \item Union: $A \cup B = \{x \mid x \in A \text{ or } x \in B\}$
    \item Intersection: $A \cap B = \{x \mid x \in A \text{ and } x \in B\}$
    \item Complement: $B \setminus A = \{x \mid x \in B \text{ and } x \notin A\}$
\end{itemize}
\end{definitionbox}

\begin{theorembox}{De Morgan's Laws}
For a family of sets $\mathcal{F}$ and a reference set $X$:
\begin{align*}
    X \setminus \left(\bigcup_{F \in \mathcal{F}} F\right) &= \bigcap_{F \in \mathcal{F}} (X \setminus F) \\
    X \setminus \left(\bigcap_{F \in \mathcal{F}} F\right) &= \bigcup_{F \in \mathcal{F}} (X \setminus F)
\end{align*}
\end{theorembox}

\subsection{Mappings and Functions}

\begin{definitionbox}{Mapping}
A mapping (or function) $f: A \to B$ is a correspondence that assigns to each $x \in A$ a unique $y \in B$, denoted $f(x)$.
\begin{itemize}
    \item Domain: The set $A$
    \item Image/Range: $f(A) = \{b \in B \mid \exists a \in A, f(a) = b\}$
\end{itemize}
\end{definitionbox}

\begin{definitionbox}{Function Types}
\begin{itemize}
    \item Onto (Surjective): $f(A) = B$
    \item One-to-One (Injective): $f(a_1) = f(a_2) \implies a_1 = a_2$
    \item Invertible (Bijective): Both one-to-one and onto
\end{itemize}
\end{definitionbox}

\begin{definitionbox}{Function Composition}
For mappings $f: A \to B$ and $g: B \to C$, the composition $g \circ f: A \to C$ is defined by $(g \circ f)(x) = g(f(x))$.
\end{definitionbox}

\subsection{Relations}

\begin{definitionbox}{Relation}
A relation $R$ on a set $X$ is a subset of $X \times X$. 
\begin{itemize}
    \item Reflexive: $\forall x \in X, xRx$
    \item Symmetric: $x R y \implies y R x$
    \item Transitive: $x R y \text{ and } y R z \implies x R z$
\end{itemize}
\end{definitionbox}

\begin{definitionbox}{Equivalence Relation}
An equivalence relation is a relation that is reflexive, symmetric, and transitive.

For an equivalence relation $R$ on $X$:
\begin{itemize}
    \item Equivalence class of $x$: $[x]_R = \{y \in X \mid xRy\}$
    \item Quotient set: $X/R = \{[x]_R \mid x \in X\}$
\end{itemize}
\end{definitionbox}

\subsection{Axiom of Choice and Zorn's Lemma}

\begin{definitionbox}{Choice Function}
A choice function on a family $\mathcal{F}$ of non-empty sets is a function $f: \mathcal{F} \to \bigcup_{F \in \mathcal{F}} F$ such that $f(F) \in F$ for all $F \in \mathcal{F}$.
\end{definitionbox}

\begin{theorembox}{Axiom of Choice}
For any non-empty collection of non-empty sets, there exists a choice function.
\end{theorembox}

\begin{definitionbox}{Partial Ordering}
A relation $\preceq$ on a set $X$ is a partial ordering if it is:
\begin{itemize}
    \item Reflexive: $x \preceq x$
    \item Antisymmetric: $x \preceq y \text{ and } y \preceq x \implies x = y$
    \item Transitive: $x \preceq y \text{ and } y \preceq z \implies x \preceq z$
\end{itemize}
\end{definitionbox}

\begin{theorembox}{Zorn's Lemma}
If a partially ordered set has an upper bound for every totally ordered subset, then it contains a maximal element.
\end{theorembox}

\begin{note}
Zorn's Lemma is equivalent to the Axiom of Choice and will be crucial in proving several important theorems in advanced mathematics.
\end{note}

\subsection{Cardinality and Set Comparisons}

\begin{definitionbox}{Cardinality}
Two sets $A$ and $B$ are said to be equipotent (or have the same cardinality) if there exists a bijective mapping $f: A \to B$. 

Key properties:
\begin{itemize}
    \item Equipotence is an equivalence relation on the collection of sets
    \item $|A| = |B|$ if and only if there exists a bijection between $A$ and $B$
    \item Finite sets have cardinality equal to the number of their elements
\end{itemize}
\end{definitionbox}

\begin{theorembox}{Cantor-Bernstein Theorem}
If there exist injective functions $f: A \to B$ and $g: B \to A$, then there exists a bijection between $A$ and $B$. 

Symbolically, if $|A| \leq |B|$ and $|B| \leq |A|$, then $|A| = |B|$.
\end{theorembox}

\subsection{Countable and Uncountable Sets}

\begin{definitionbox}{Countable Sets}
A set $A$ is:
\begin{itemize}
    \item Finite if $|A| < \aleph_0$
    \item Countably infinite if $|A| = \aleph_0$
    \item Countable if it is either finite or countably infinite
    \item Uncountable if it is not countable
\end{itemize}
\end{definitionbox}

\begin{theorembox}{Countability Properties}
\begin{itemize}
    \item The set of natural numbers $\mathbb{N}$ is countably infinite
    \item The set of integers $\mathbb{Z}$ is countably infinite
    \item The set of rational numbers $\mathbb{Q}$ is countably infinite
    \item The set of real numbers $\mathbb{R}$ is uncountable
\end{itemize}
\end{theorembox}

\subsection{Generalized Cartesian Product}

\begin{definitionbox}{Generalized Cartesian Product}
For a family of sets $\{E_{\lambda}\}_{\lambda\in\Lambda}$ indexed by $\Lambda$, the Cartesian product is defined as:

$$\prod_{\lambda\in\Lambda} E_{\lambda} = \left\{f: \Lambda \to \bigcup_{\lambda\in\Lambda} E_{\lambda} \mid \forall \lambda \in \Lambda, f(\lambda) \in E_{\lambda}\right\}$$
\end{definitionbox}

\begin{theorembox}{Cartesian Product Properties}
\begin{itemize}
    \item If $\Lambda$ is finite and each $E_{\lambda}$ is non-empty, then $\prod_{\lambda\in\Lambda} E_{\lambda}$ is non-empty
    \item The Axiom of Choice is equivalent to the statement that the Cartesian product of a non-empty family of non-empty sets is non-empty
\end{itemize}
\end{theorembox}

\subsection{Well-Ordering Principle}

\begin{theorembox}{Well-Ordering Theorem}
Every set can be well-ordered. That is, for every set $X$, there exists a total order $\leq$ on $X$ such that every non-empty subset of $X$ has a least element.

This theorem is equivalent to the Axiom of Choice.
\end{theorembox}

\subsection{Foundations of Set Theory}

\begin{definitionbox}{Zermelo-Fraenkel Axioms}
Set theory is formally grounded in the Zermelo-Fraenkel (ZF) axiom system, which provides a rigorous foundation for mathematical reasoning about sets:

\begin{itemize}
    \item \textbf{Axiom of Extensionality:} Two sets are equal if and only if they have exactly the same elements
    \item \textbf{Axiom of Pairing:} For any two sets, there exists a set containing exactly those two sets
    \item \textbf{Axiom of Union:} For any collection of sets, there exists a set that contains all elements that belong to at least one set in the collection
    \item \textbf{Axiom of Power Set:} For any set, there exists a set containing all possible subsets of that set
\end{itemize}

The Axiom of Choice (AC) can be added to ZF to form ZFC, the standard foundation for most mathematical reasoning.
\end{definitionbox}

\begin{definitionbox}{Parametrization of Sets}
For a set $\Lambda$ and a family of sets $\{E_{\lambda}\}_{\lambda\in\Lambda}$, the parametrization provides a systematic way to index sets:

\begin{itemize}
    \item $\Lambda$ is called the \textit{index set}
    \item Each $E_{\lambda}$ is associated with a unique $\lambda \in \Lambda$
    \item Different parametrizations of the same family may yield different mathematical structures
\end{itemize}

Example: Let $\Lambda = \{1,2\}$ and define $E_1 = \{a,b\}$, $E_2 = \{c,d\}$. Then $\{E_{\lambda}\}_{\lambda\in\Lambda} = \{\{a,b\}, \{c,d\}\}$.
\end{definitionbox}

\begin{theorembox}{Parametrized Cartesian Product Properties}
For a parametrized family $\{E_{\lambda}\}_{\lambda\in\Lambda}$:

\begin{itemize}
    \item The Cartesian product depends critically on the specific parametrization
    \item Two different indexings of the same underlying sets can produce different Cartesian products
    \item The choice of index set $\Lambda$ is crucial in defining the product structure
\end{itemize}
\end{theorembox}

\begin{note}
The subtleties of set parametrization highlight the depth and complexity inherent in foundational set theory, demonstrating how seemingly simple concepts can reveal profound mathematical structures.
\end{note}

\end{document}
